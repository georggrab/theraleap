\chapter{Introduction}
\section{Problem description}
In 2016, approximately 60.000 german residents have been hospitalized due to hand or wrist injuries \cite{DeStatisHandInjuries}. The initial treatment of such injuries is often followed by a lengthy recovery phase in which ergotherapeutic treatment occurs in order to further aid recovery. As ergo therapy sessions are usually held in one-on-one sessions, this results in a significant time and resource both by the patient and the treating clinic. Additionally, the sessions themselves are often described by patients as boring and unmotivating by their repetitive nature.

Ergo therapists of the Katharinenhospital Stuttgart are currently researching alternative treatment methods that could be an improvement to all three of these fundamental problems. The aim of the research is to introduce various gamification aspects to the recovery sessions. The patients should be enabled to do repetitive parts of the recovery exercises at home by means of successfully executing them while controlling video games. This methodology of executing prevention and rehabilitation measures is an active and well known field of research commonly referred to as Exergames\cite{RehaCareExergames}. In a more general sense, games that are designed with a second primary purpose (apart from entertainment) are referred to as serious games \cite{SeriousGamesBook}. 

\section{Project Scope}
The scope of this student research project is to design and provide implementations for a software solution acting as an underlying framework on which the Exergames can be executed. The framework should be capable of meeting the determined requirements as outlined in section \ref{sec:reqanalysis} either directly through the provided implementation, or, if some requirements cannot be fulfilled with the reference implementation, by way of easy extensibility. The framework should also contain a user facing component where relevant measurements and game configurations can be made and from which the games are executed. Additionally, the framework should contain various tools that will make it easier for future developers to develop and debug subcomponents.

\section{Requirements Analysis}
\label{sec:reqanalysis}
Outsourcing recovery exercises into a space where no direct therapeutic supervision is available generates a series of challenges that have to be identified and overcome before successfully integrating Exergames in the recovery sessions. 

\subsection{Functional Requirements}
In a software development context, the challenges a system has to solve in order to become useful to the stakeholders are referred to as functional requirements.  The most notable functional requirements are outlined as follows.

\subsubsection{Domain Virtualization}
In order for Exergames to fundamentally function, they require an accurate, real-time virtualized representation of the problem domain. For example, in order to develop Exergames for treating hand injuries, a virtual representation of the hand must be available.

\subsubsection{Exercise Classification}
\label{sec:exercise-classification}
The most important capability of an Exergame is to correctly classify whether a recovery exercise has been executed. In the domain of hand and wrist injury recovery, a recovery exercise may for example be the spreading of the thumb, where the remaining fingers of the hand remain closed. Other examples for recovery exercises have been outlined by \cite{StudiArbeitVolzBaumotte}. The result of the exercise classification can then be used as a gameplay element in the Exergame. For example, if a thumb spread exercise as described above has been executed well enough, a \emph{Space Invaders-like} Exergame could trigger the space ship to shoot.

\subsubsection{Patient Adaptibility}
One-on-one therapy sessions in ergo therapy are required because of the large variety of different hand injuries, each requiring a different set of recovery exercises. Additionally, the classification logic (see \ref{sec:exercise-classification}) for the recovery exercises themselves have to be adaptable to how far the patient has progressed so far in recovery. For example, if the patient is progressing well in recovery, the relevant exercise has to be increased in difficulty in order for the treatment to remain effective.

\subsubsection{Monitorability}
The ergo therapist has to be able to view monitoring information related to the patients playing activity. Most fundamentally, the therapist should be able to view the number of times and total duration of Exergames played in order to verify if the agreed upon exercise volume has been completed. Additionally, specific information that aid the ergo therapist in assessing the recovery progress of the patient should be available. If the therapist determines that the current exercise has to be adapted in some way, or for exchanging other kinds of information with the patient, such as providing hints or agreeing on the next physical appointment date, this should be possible through an integrated messaging platform.

\subsubsection{Gameplay and Frontend}
\label{sec:gameplay}
Finally, the system should provide a frontend component, from which the actual games are executed and configured, and where display components relevant for resolving other software requirements can be found.

\subsection{Non-Functional Requirements}
In addition to the functional requirements, the following \glspl{NFR} have to be considered while designing and implementing the system. \glspl{NFR} are global requirements that are not directly related to function, but refer to the development or operational costs of the system, such as performance, reliability, and maintainability \cite{chung2012non}.

\subsubsection{Modularization}
On a technical level, the program logic responsible for classifying if an exercise has been completed (see \ref{sec:exercise-classification}) should be separated from the actual Exergames logic (see \ref{sec:gameplay}). This would pose the advantage of introducing a modular aspect to the system, as both exercise classifiers and games could be exchanged, both keeping the patients engaged in the platform by allowing them to train their assigned exercise using a variety of games, and greatly simplifying the work of future developers, as they will be enabled to develop games for the platform without any prior knowledge of exercise classification, and vice versa.

\subsubsection{Performance}
Performance is a critical \gls{NFR} for the system. All data coming from the device providing the relevant domain virtualization has to be ingested, preprocessed, and classified in real time. If this is not the case, the patient will experience a significant lag between the performed exercise and the feedback of the Exergames, quickly resulting in frustration. Additionally, the execution of the Exergames themselves should be performant enough so that the gameplay experience isn't negatively obstructed. 

\subsubsection{Ease of deployment}
The system should ultimately be primarily executed on a patient provided device. As such, deployment of the application should be easy, and robust with respect to a multitude of possible, previously unknown target environments.

\subsubsection{Extensibility}
As the system is acting primarily as an underlying framework on which other developers should build upon in the future, it should be written in a way that allows for easy extensibility. It should especially be written in a computer programming language that is well known to the potential target developer audience, so minimum prior knowledge is required before starting development with the project. Additionally, the framework should be future proof: it should be simple to exchange subcomponents with more modern equivalents in the future. For example, it should be simple to add support for more modern hardware devices providing domain virtualization, or more modern graphics libraries for developing the Exergames in the future.

\section{Solution Design}
\subsection{Available Alternatives}
"dumb" web platform for visualizing, main work happening in server processes running locally

fully featured web platform, doing everything

GUI Application
\subsection{Elected Alternative}
fully featured web platform, because modern web technologies allow it, GUI Application does not fit the requirements, and separating main work into server thread is, while potentially more performant, both potentially insecure and unnecessarily complex.